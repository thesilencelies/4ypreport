 \section{Magic: the Gathering}
  The particular game type this project was chosen for was collectible card games, in particular starting with Magic: the Gathering (MtG). These games provide a number of interesting and difficult challenges for AI: uncertain information, stochastic results, variable action spaces, along with additional opportunity for further depth should deck building also be considered. They are also turn based and don't require a physics engine, so they can run through many iterations of play quickly. Of these, MtG was chosen as it represents both a significant breadth of possibilities and different interactions without excessive card complexity or specificity and it has a more approachable learning curve than most. Hearthstone was considered, but a suitable emulator was hard to find due to the copyright issues. 
  \subsection{Aims}
  This project aims to produce an architecture that can learn to play at or around human competency trading card games. Ideally this would involve use of novel techniques that could find other use as well.
  \subsection{Goals}
  \begin{itemize}
    \item Produce an architecture that can learn to play MtG to an acceptable level of competency with a given deck
    \item Produce an architecture, possibly using the above architecture, that can produce and improve decks of it's own and learn to adapt to changes in metagame.
    \item generalise this architecture to be able to learn and play other card games
  \end{itemize}
  \section{Completed Work}
    \subsection{Initial Setup}
      A suitable open source emulation environment for playing MtG was identified, and modifications were made to it to allow the learned AI to be used in it and trained agains the extant rules based AI. Neural Networks libraries for java that use the GPU were installed and unit tested. An overview of AI techniques and the particulars of standard reinforcement learning algorithms were read. The exact state size was considered, and some additional restrictions were made on the type of cards that would be used within MtG's 15,000 card pool to reduce the initial complexity.
    \subsection{Preliminary Development}
      As an initial test of the neural network architecture with reinforcement learning, a simple maze enviroment was made. Currently the agent is not converging to a solution, and so the details of the implimentation are being looked at.

  \section{Initial Changes}
  To get a firmer grip on reinforcement learning techniques, the maze agent will be tested with a number of tabular methods as well as the current function approximation basede technique.Then the structure for the preliminary reinforcement learning agent to play MtG will be made. Initially it will only choose what card to play, then other play decisions. It's likely a na\"\i eve approach won't be sufficient, so a number of possible improvements are on the boards, with more to be considered:

\section{Final Assessment of Feasibility}
